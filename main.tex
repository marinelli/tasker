\documentclass{beamer}
%
% Choose how your presentation looks.
%
% For more themes, color themes and font themes, see:
% http://deic.uab.es/~iblanes/beamer_gallery/index_by_theme.html
%
\mode<presentation>
{
  \usetheme{default}      % or try Darmstadt, Madrid, Warsaw, ...
  \usecolortheme{default} % or try albatross, beaver, crane, ...
  \usefonttheme{default}  % or try serif, structurebold, ...
  \setbeamertemplate{navigation symbols}{}
  \setbeamertemplate{caption}[numbered]
} 

\usepackage[english]{babel}
\usepackage[utf8]{inputenc}
\usepackage[T1]{fontenc}
\usepackage{verbatim}
\usepackage{tabto}
\usepackage{pmboxdraw}
\usepackage{textcomp}
\usepackage{fancyvrb,newverbs,xcolor}

\title{Functional logic programming}
\author{Giorgio Marinelli\thanks{\url{https://giorgiomarinelli.it/cv}}}
% \date{}

\addtobeamertemplate{navigation symbols}{}{
  \usebeamerfont{footline}
  \usebeamercolor[fg]{footline}
  \hspace{1em}
  \insertframenumber
}

\definecolor{cverbbg}{gray}{0.93}

\newenvironment{cverbatim}
 {\SaveVerbatim{cverb}}
 {\endSaveVerbatim
  \flushleft\fboxrule=0pt\fboxsep=.5em
  \colorbox{cverbbg}{\BUseVerbatim{cverb}}%
  \endflushleft
}

\newenvironment{lcverbatim}
 {\SaveVerbatim{cverb}}
 {\endSaveVerbatim
  \flushleft\fboxrule=0pt\fboxsep=.5em
  \colorbox{cverbbg}{%
    \makebox[\dimexpr\linewidth-2\fboxsep][l]{\BUseVerbatim{cverb}}%
  }
  \endflushleft
}

\newcommand{\ctexttt}[1]{\colorbox{cverbbg}{\texttt{#1}}}
\newverbcommand{\cverb}
  {\setbox\verbbox\hbox\bgroup}
  {\egroup\colorbox{cverbbg}{\box\verbbox}}

\begin{document}

% % %

\begin{frame}
  \titlepage
\end{frame}

% % %

\section{Introduction}
\begin{frame}{Introduction}
Functional logic programming is the combination of two kind of declarative languages:
\begin{itemize}
  \item Functional languages (eg. LISP, ML, Haskell, ...)
  \item Logic languages (eg. Prolog, Datalog, ...)
\end{itemize}

\end{frame}

% % %

\section{Introduction}
\begin{frame}{Introduction}

An example of a functional logic programming language is Curry\footnote{\url{http://www.curry-language.org/}}

\vspace{1em}

It is based on the Haskell language:

\vspace{1em}

\begin{itemize}
  \item Statically typed
  \item Purely functional
  \item Type inference
  \item Lazy evaluation
\end{itemize}

\end{frame}

% % %

\section{Introduction}
\begin{frame}{Introduction}

It has \emph{functional logic programming} features, like:

\vspace{1em}

\begin{itemize}
  \item Narrowing
  \item Functional patterns
  \item Non-determinism
  \item Strategies
\end{itemize}

\vspace{1em}

There exist two main implementation of the language:

\vspace{1em}

\begin{itemize}
  \item PAKCS: compile Curry programs into Prolog programs
  \item KiCS2: compile Curry programs into Haskell programs
\end{itemize}

\end{frame}

% % %

\section{Narrowing}
\begin{frame}[fragile]{Narrowing}

One of the main feature in the Curry language is Narrowing: a combination of variable instantiation and term reduction (originally introduced in automated theorem proving).

\begin{cverbatim}
last :: [a] -> a
last xs | zs ++ [e] =:= xs   =   e
  where zs, e free
\end{cverbatim}

\end{frame}

% % %

\section{Functional pattern}
\begin{frame}[fragile]{Functional pattern}

The \ctexttt{last} rule can be reformulated using \emph{functional pattern}: a pattern with a function inside.

\begin{cverbatim}
last :: [a] -> a
last (zs ++ [e]) = e
\end{cverbatim}

\vspace{1em}

Haskell, do not allow this rule because it is not constructor based.

\end{frame}

% % %

\section{Non-determinism}

\begin{frame}[fragile]{Non-determinism}

Curry allow to define non deterministic computations (or operations).

\vspace{1em}

\begin{cverbatim}
(?) :: a -> a -> a
x ? y = x
x ? y = y
\end{cverbatim}

\vspace{1em}

We can define an non deterministic operation like flipping a coin:

\begin{cverbatim}
coin :: Int
coin = 0 ? 1
\end{cverbatim}

\end{frame}

% % %

\section{Demo}

\begin{frame}[c]{}

\centering {\huge Demo}

\end{frame}

\end{document}
